%
% MUSCLE SmartCard Development (http://www.musclecard.com)
%
% Copyright (C) 2004
%  David Corcoran <corcoran@musclecard.com>
%  Ludovic Rousseau <ludovic.rousseau@free.fr>
%
% $Id$

\documentclass[a4paper,12pt]{article}

\usepackage{longtable}
\usepackage{url}
\usepackage{varioref}

% D�tection de pdflatex
\ifx\pdfcompresslevel\undefined
% Si on fait un postscript
\typeout{Postscript version}
\usepackage[dvips]{graphicx,rotating}
%\usepackage[dvips,matrix,line,curve,arrow,frame]{xy}
\DeclareGraphicsExtensions{.eps}
\else
% Si on fait un PDF
\typeout{PDF version}
\usepackage[pdftex]{graphicx,rotating}
%\usepackage[matrix,line,curve,arrow,frame]{xy}
\DeclareGraphicsExtensions{.jpg,.pdf}
\usepackage[pdftex]{hyperref}
\usepackage{ae,aeguill}
\fi

% smaller margins
\usepackage{fullpage}

% do not number subsubsection
\setcounter{tocdepth}{2}
\setcounter{secnumdepth}{2}

\newcommand{\tab}{}
\newcommand{\synopsis}{\subsubsection{Synopsis:}}
\newcommand{\parameters}{\subsubsection{Parameters:}}
\newcommand{\desc}{\subsubsection{Description:}}
\newcommand{\example}{\subsubsection{Example:}}
\newcommand{\returns}{\subsubsection{Returns:}}

\title{MUSCLE PC/SC IFD Driver API}
\author{David Corcoran \& Ludovic Rousseau\\
\url{corcoran@musclecard.com}, \url{ludovic.rousseau@free.fr}}
\date{July 28, 2004}

\begin{document}

\maketitle

\begin{abstract}
This toolkit and documentation is provided on an as is basis. The
authors shall not be held responsible for any mishaps caused by the use
of this software.

For more information please visit \url{http://www.musclecard.com/}.

\end{abstract}

% space between paragraphs
\parskip = 8pt

% remove paragraph indentation
\addtolength{\parindent}{-\parindent}

Document history: \\
\begin{tabular}{|l|l|l|}
\hline
3.0.1 & August 9, 2003 & latest PDF only version \\
\hline
3.1.0 & July 28, 2004 & reformat using \LaTeX{}, correct bugs and add
information \\
\hline
\end{tabular}

\newpage
\tableofcontents
\newpage

%---------%---------%---------%---------%---------%---------%---------
\section{Introduction/Overview}

This document describes the API calls required to make a PC/SC driver
for a device to be supported under the MUSCLE PC/SC resource manager. By
implementing these calls correctly in a driver or shared object form,
reader manufacturers can fit their hardware into an already existing
infrastructure under several operating systems and hardware platforms.
This IFD Handler interface is not restricted to smart cards and readers
and could also be used for other types of smart card like devices. I
would really like to hear from you. If you have any feedback either on
this documentation or on the MUSCLE project please feel free to email me
at: \url{corcoran@musclecard.com}.



%---------%---------%---------%---------%---------%---------%---------
\section{Definitions}


%---------%---------%---------%---------%---------%---------
\subsection{Defined types}

The following is a list of commonly used type definitions in the
following API. These definitions and more can be found in the
\texttt{ifdhandler.h} file.

{\tt
\begin{longtable}{|l|l|}
\hline
\textrm{PC/SC type} & \textrm{C type} \\
\hline
\hline
DWORD & unsigned long \\
LPSTR & char * \\
PDWORD & unsigned long * \\
PUCHAR & unsigned char * \\
RESPONSECODE & long \\
VOID & void \\
\hline
\end{longtable}
}


%---------%---------%---------%---------%---------%---------
\subsection{Error codes}

The following is a list of returned values:

{\tt
\begin{longtable}{|l|}
\hline
IFD\_SUCCESS \\
\hline
IFD\_COMMUNICATION\_ERROR\\
IFD\_ERROR\_CONFISCATE\\
IFD\_ERROR\_EJECT\\
IFD\_ERROR\_NOT\_SUPPORTED\\
IFD\_ERROR\_POWER\_ACTION\\
IFD\_ERROR\_PTS\_FAILURE\\
IFD\_ERROR\_SET\_FAILURE\\
IFD\_ERROR\_SWALLOW\\
IFD\_ERROR\_TAG\\
IFD\_ERROR\_VALUE\_READ\_ONLY\\
IFD\_ICC\_NOT\_PRESENT\\
IFD\_ICC\_PRESENT\\
IFD\_NOT\_SUPPORTED\\
IFD\_PROTOCOL\_NOT\_SUPPORTED\\
IFD\_RESPONSE\_TIMEOUT\\
\hline
\end{longtable}
}

%---------%---------%---------%---------%---------%---------%---------
\section{Readers configuration}


%---------%---------%---------%---------%---------%---------
\subsection{USB readers}

USB readers use the bundle approach so that the reader can be loaded and
unloaded upon automatic detection of the device. The bundle approach is
simple: the actual library is just embedded in a directory so additional
information can be gathered about the device.

A bundle looks like the following:

\begin{verbatim}
GenericReader.bundle/
  Contents/
    Info.plist  - XML file describing the reader
    MacOS/      - Driver directory for OS X
    Solaris/    - Driver directory for Solaris
    Linux/      - Driver directory for Linux
    HPUX/       - Driver directory for HPUX
\end{verbatim}

The \texttt{Info.plist} file describes the driver and gives the loader all
the necessary information. The following must be contained in the
\texttt{Info.plist} file:

\begin{itemize}
\item \texttt{ifdVendorID}

The vendor ID of the USB device.

Example:
\begin{verbatim}
<key>ifdVendorID</key>
<string>0x04E6</string>
\end{verbatim}

You may have an OEM of this reader in which an additional
\texttt{<string>} can be used like in the below example:
\begin{verbatim}
<key>ifdVendorID</key>
<array>
  <string>0x04E6</string>
  <string>0x0973</string>
</array>
\end{verbatim}

If multiples exist all the other parameters must have a second value
also. You may chose not to support this feature but it is useful when
reader vendors OEM products so you only distribute one driver.

The CCID driver from Ludovic
Rousseau\footnote{\url{http://pcsclite.alioth.debian.org/ccid.html}}
uses this feature since the same driver supports many different readers.

\item \texttt{ifdProductID}

The product id of the USB device.

\begin{verbatim}
<key>ifdProductID</key>
<string>0x3437</string>
\end{verbatim}

\item \texttt{ifdFriendlyName}

Example:
\begin{verbatim}
<key>ifdFriendlyName</key>
<string>SCM Microsystems USB Reader</string>
\end{verbatim}

\item \texttt{CFBundleExecutable}

The executable name which exists in the particular platform's directory.

Example:
\begin{verbatim}
<key>CFBundleExecutable</key>
<string>libccid.so.0.4.2</string>
\end{verbatim}

\end{itemize}

Complete sample file:
\begin{verbatim}
<?xml version="1.0" encoding="UTF-8"?>
<!DOCTYPE plist PUBLIC "-//Apple Computer//DTD PLIST 1.0//EN"
    "http://www.apple.com/DTDs/PropertyList-1.0.dtd">
<plist version="1.0">
<dict>
    <key>CFBundleDevelopmentRegion</key>
    <string>English</string>
    <key>CFBundleInfoDictionaryVersion</key>
    <string>6.0</string>
    <key>CFBundlePackageType</key>
    <string>BNDL</string>
    <key>CFBundleSignature</key>
    <string>????</string>
    <key>CFBundleVersion</key>
    <string>0.0.1d1</string>
    <key>ifdCapabilities</key>
    <string>0x00000000</string>
    <key>ifdProtocolSupport</key>
    <string>0x00000001</string>
    <key>ifdVersionNumber</key>
    <string>0x00000001</string>

    <key>CFBundleExecutable</key>
    <string>libfoobar.so.x.y</string>

    <key>ifdManufacturerString</key>
    <string>Foo bar inc.</string>

    <key>ifdProductString</key>
    <string>Driver for Foobar reader, version x.y</string>

    <key>ifdVendorID</key>
    <string>0x1234</string>

    <key>ifdProductID</key>
    <string>0x5678</string>

    <key>ifdFriendlyName</key>
    <string>Foobar USB reader</string>
</dict>
</plist>
\end{verbatim}

As indicated in the XML file the DTD is available at
\url{http://www.apple.com/DTDs/PropertyList-1.0.dtd}.


%---------%---------%---------%---------%---------%---------
\subsection{Serial readers}

Serial drivers must be configured to operate on a particular port and
respond to a particular name. The \texttt{reader.conf} file is used for
this purpose.

It has the following syntax:

\begin{verbatim}
# Configuration file for pcsc-lite
# David Corcoran <corcoran@musclecard.com>

FRIENDLYNAME  Generic Reader
DEVICENAME    /dev/ttyS0
LIBPATH       /usr/lib/pcsc/drivers/libgen_ifd.so
CHANNELID     1
\end{verbatim}

\begin{itemize}
\item The pound sign \verb+#+ denotes a comment.

\item The \texttt{FRIENDLYNAME} field is an arbitrary text used to
identify the reader. This text is displayed by commands like
\texttt{pcsc\_scan}\footnote{\url{http://ludovic.rousseau.free.fr/softwares/pcsc-tools/}}
that prints the names of all the connected and detected readers.

\item The \texttt{DEVICENAME} field was not used for old drivers (using
the IFD handler version 2.0 or previous). It is now (IFD handler version
3.0) used to identify the physical port on which the reader is
connected. This is the device name of this port. It is dependent of the
OS kernel. For example the first serial port device is called
\texttt{/dev/ttyS0} under Linux and \texttt{/dev/cuaa0} under FreeBSD.

\item The \texttt{LIBPATH} field is the filename of the driver code. The
driver is a dynamically loaded piece of code (generally a
\texttt{drivername.so*} file).

\item The \texttt{CHANNELID} is no more used for recent drivers (IFD
handler 3.0) and has been superseded by \texttt{DEVICENAME}. If you
have an old driver this field is used to indicate the port to use. You
should read your driver documentation to know what information is needed
here. It should be the serial port number for a serial reader.

\texttt{CHANNELID} was the numeric version of the port in which the
reader will be located. This may be done by a symbolic link where
\texttt{/dev/pcsc/1} is the first device which may be a symbolic link to
\texttt{/dev/ttyS0} or whichever location your reader resides.

\end{itemize}


%---------%---------%---------%---------%---------%---------%---------
\section{API Routines}

The routines specified hereafter will allow you to write an IFD handler
for the PC/SC Lite resource manager. Please use the complement
developer's kit complete with headers and Makefile at:
\url{http://www.musclecard.com/drivers.html}.

This gives a common API for communication to most readers in a
homogeneous fashion. This document assumes that the driver developer is
experienced with standards such as ISO-7816-(1, 2, 3, 4), EMV and MCT
specifications. For listings of these specifications please access the
above web site.


%---------%---------%---------%---------%---------%---------
\subsection{IFDHCreateChannel}

\synopsis

\begin{verbatim}
#include <PCSC/ifdhandler.h>

RESPONSECODE IFDHCreateChannel(DWORD Lun,
    DWORD Channel);
\end{verbatim}

\parameters

\begin{tabular}{lll}
\texttt{Lun} & IN & Logical Unit Number \\
\texttt{Channel} & IN & Channel ID \\
\end{tabular}

\desc

This function is required to open a communications channel to the port
listed by \texttt{Channel}. For example, the first serial reader on COM1
would link to \texttt{/dev/pcsc/1} which would be a symbolic link to
\texttt{/dev/ttyS0} on some machines This is used to help with
inter-machine independence.

On machines with no \texttt{/dev} directory the driver writer may choose to
map their \texttt{Channel} to whatever they feel is appropriate.

Once the channel is opened the reader must be in a state in which it is
possible to query \texttt{IFDHICCPresence()} for card status.

\begin{itemize}
\item \texttt{Lun} - Logical Unit Number

Use this for multiple card slots or multiple readers.
\texttt{0xXXXXYYYY} - \texttt{XXXX} multiple readers, \texttt{YYYY}
multiple slots. The resource manager will set these automatically. By
default the resource manager loads a new instance of the driver so if
your reader does not have more than one smart card slot then ignore the
Lun in all the functions.

PC/SC supports the loading of multiple readers through one instance of
the driver in which \texttt{XXXX} is important. \texttt{XXXX} identifies
the unique reader in which the driver communicates to. The driver should
set up an array of structures that associate this \texttt{XXXX} with the
underlying details of the particular reader.

\item \texttt{Channel} - Channel ID

This is denoted by the following:

{\tt
\begin{tabular}{ll}
0x000001 & /dev/pcsc/1\\
0x000002 & /dev/pcsc/2\\
0x000003 & /dev/pcsc/3\\
0x000004 & /dev/pcsc/4\\
\end{tabular}
}

USB readers can ignore the \texttt{Channel} parameter and query the USB
bus for the particular reader by manufacturer and product id.

\end{itemize}

\returns

\begin{tabular}{ll}
\texttt{IFD\_SUCCESS} & Successful\\
\texttt{IFD\_COMMUNICATION\_ERROR} & Error has occurred
\end{tabular}


%---------%---------%---------%---------%---------%---------
\subsection{IFDHCreateChannelByName}
\label{IFDHCreateChannelByName}

\synopsis
\begin{verbatim}
#include <PCSC/ifdhandler.h>

RESPONSECODE IFDHCreateChannelByName(DWORD Lun,
    LPSTR deviceName);
\end{verbatim}

\parameters

\begin{tabular}{lll}
\texttt{Lun} & IN & Logical Unit Number \\
\texttt{DeviceName} & IN & String device path \\
\end{tabular}


\desc

This function is required to open a communications channel to the port
listed by \texttt{DeviceName}.

Once the channel is opened the reader must be in a state in which it is
possible to query \texttt{IFDHICCPresence()} for card status.

\begin{itemize}
\item \texttt{Lun} - Logical Unit Number

Use this for multiple card slots or multiple readers.
\texttt{0xXXXXYYYY} - \texttt{XXXX} multiple readers, \texttt{YYYY}
multiple slots. The resource manager will set these automatically. By
default the resource manager loads a new instance of the driver so if
your reader does not have more than one smart card slot then ignore the
Lun in all the functions.

PC/SC supports the loading of multiple readers through one instance of
the driver in which \texttt{XXXX} is important. \texttt{XXXX} identifies
the unique reader in which the driver communicates to. The driver should
set up an array of structures that associate this \texttt{XXXX} with the
underlying details of the particular reader.

\item \texttt{DeviceName} - filename to use by the driver.

For drivers configured by \texttt{/etc/reader.conf} this is the value of
the field \texttt{DEVICENAME}.

For USB drivers under platforms using
\texttt{libusb}\footnote{\url{http://libusb.sourceforge.net/}} for USB
abstraction (Any Unix except MacOSX) the \texttt{DeviceName} field uses
the string generated by:
\begin{verbatim}
printf("usb:%04x/%04x:libusb:%s:%s",
    idVendor, idProduct,
    bus->dirname, dev->filename)
\end{verbatim}

So it is something like: \texttt{usb:08e6/3437:libusb:001:042} under
Linux.

It is the responsability of the driver to correctly identify the reader.
This scheme was put in place to be able to distinguish two identical
readers connected at the same time.

\end{itemize}

\returns

\begin{tabular}{ll}
\texttt{IFD\_SUCCESS} & Successful\\
\texttt{IFD\_COMMUNICATION\_ERROR} & Error has occurred
\end{tabular}


%---------%---------%---------%---------%---------%---------
\subsection{IFDHCloseChannel}

\synopsis
\begin{verbatim}
#include <PCSC/ifdhandler.h>

RESPONSECODE IFDHCloseChannel(DWORD Lun);
\end{verbatim}

\parameters

\begin{tabular}{lll}
\texttt{Lun} & IN & Logical Unit Number \\
\end{tabular}

\desc

This function should close the reader communication channel for the
particular reader. Prior to closing the communication channel the reader
should make sure the card is powered down and the terminal is also
powered down.

\returns

\begin{tabular}{ll}
\texttt{IFD\_SUCCESS} & Successful\\
\texttt{IFD\_COMMUNICATION\_ERROR} & Error has occurred
\end{tabular}


%---------%---------%---------%---------%---------%---------
\subsection{IFDHGetCapabilities}

\synopsis
\begin{verbatim}
#include <PCSC/ifdhandler.h>

RESPONSECODE IFDHGetCapabilities(DWORD Lun,
    DWORD Tag,
    PDWORD Length,
    PUCHAR Value);
\end{verbatim}

\parameters

\begin{tabular}{lll}
\texttt{Lun} & IN & Logical Unit Number \\
\texttt{Tag} & IN & Tag of the desired data value \\
\texttt{Length} & INOUT & Length of the desired data value \\
\texttt{Value} & OUT & Value of the desired data \\
\end{tabular}

\desc

This function should get the slot/card capabilities for a particular
slot/card specified by \texttt{Lun}. Again, if you have only 1 card slot
and don't mind loading a new driver for each reader then ignore
\texttt{Lun}.

\begin{itemize}
\item \texttt{Tag} - the tag for the information requested

\begin{itemize}
\item \texttt{TAG\_IFD\_ATR}

Return the ATR and it's size (implementation is mandatory).

\item \texttt{SCARD\_ATTR\_ATR\_STRING}

Same as \texttt{TAG\_IFD\_ATR} but this one is not mandatory. It is
defined in Microsoft PC/SC \texttt{SCardGetAttrib()}.

\item \texttt{TAG\_IFD\_SIMULTANEOUS\_ACCESS}

Return the number of sessions (readers) the driver can handle in
\texttt{Value[0]}.

This is used for multiple readers sharing the same driver.

\item \texttt{TAG\_IFD\_THREAD\_SAFE}

If the driver support more than one reader (see
\texttt{TAG\_IFD\_SIMULTANEOUS\_ACCESS} above) this tag indicates if the
driver supports access to multiple reader at the same time.

\texttt{Value[0] = 1} indicates the driver supports simultaneous
accesses.

\item \texttt{TAG\_IFD\_SLOTS\_NUMBER}

Return the number of slots in this reader in \texttt{Value[0]}.

\item \texttt{TAG\_IFD\_SLOT\_THREAD\_SAFE}

If the reader has more than one slot (see
\texttt{TAG\_IFD\_SLOTS\_NUMBER} above) this tag indicates if the driver
supports access to multiple slots of the same reader at the same time. 

\texttt{Value[0] = 1} indicates the driver supports simultaneous slot
accesses.

\item \texttt{IOCTL\_SMARTCARD\_VENDOR\_VERIFY\_PIN}

Return a non null value if the driver/reader is able to perform a
\texttt{SCardControl(hCard, IOCTL\_SMARTCARD\_VENDOR\_VERIFY\_PIN,
...)}. See pcsc-lite documentation~\cite{pcsclite}.

\end{itemize}

\item \texttt{Length} - the length of the returned data

\item \texttt{Value} - the value of the data

\end{itemize}

This function is also called when the application uses the PC/SC
\texttt{SCardGetAttrib()} function. The list of supported tags is not
limited.  The ones above are used by the PC/SC lite resource manager.

\returns

\begin{tabular}{ll}
\texttt{IFD\_SUCCESS} & Successful\\
\texttt{IFD\_ERROR\_TAG} & Invalid tag given
\end{tabular}


%---------%---------%---------%---------%---------%---------
\subsection{IFDHSetCapabilities}

\synopsis
\begin{verbatim}
#include <PCSC/ifdhandler.h>

RESPONSECODE IFDHSetCapabilities(DWORD Lun,
    DWORD Tag,
    DWORD Length,
    PUCHAR Value);
\end{verbatim}

\parameters

\begin{tabular}{lll}
\texttt{Lun} & IN & Logical Unit Number\\
\texttt{Tag} & IN & Tag of the desired data value\\
\texttt{Length} & INOUT & Length of the desired data value\\
\texttt{Value} & OUT & Value of the desired data\\
\end{tabular}

\desc

This function should set the slot/card capabilities for a particular
slot/card specified by Lun. Again, if you have only 1 card slot and
don't mind loading a new driver for each reader then ignore Lun.

\begin{itemize}
\item \texttt{Tag} - the tag for the information needing set

\begin{itemize}
\item \texttt{TAG\_IFD\_SLOTNUM}

This is used in IFDHandler v1.0 to select the slot to use for the next
\texttt{IFD\_*} command. This tag is no more used with versions 2.0 and
3.0 of the IFD Handler.

\end{itemize}

\item \texttt{Length} - the length of the data
\item \texttt{Value} - the value of the data
\end{itemize}

This function is also called when the application uses the PC/SC
\texttt{SCardGetAttrib()} function. The list of supported tags is not limited.

\returns

\begin{tabular}{ll}
\texttt{IFD\_SUCCESS} & Success\\
\texttt{IFD\_ERROR\_TAG} & Invalid tag given\\
\texttt{IFD\_ERROR\_SET\_FAILURE} & Could not set value\\
\texttt{IFD\_ERROR\_VALUE\_READ\_ONLY} & Trying to set read only value
\end{tabular}


%---------%---------%---------%---------%---------%---------
\subsection{IFDHSetProtocolParameters}

\synopsis
\begin{verbatim}
#include <PCSC/ifdhandler.h>

RESPONSECODE IFDHSetProtocolParameters(DWORD Lun,
    DWORD Protocol,
    UCHAR Flags,
    UCHAR PTS1,
    UCHAR PTS2,
    UCHAR PTS3);
\end{verbatim}

\parameters

\begin{tabular}{lll}
\texttt{Lun} & IN & Logical Unit Number\\
\texttt{Protocol} & IN & Desired protocol\\
\texttt{Flags} & IN & OR'd Flags (See below)\\
\texttt{PTS1} & IN & 1st PTS Value\\
\texttt{PTS2} & IN & 2nd PTS Value\\
\texttt{PTS3} & IN & 3rd PTS Value\\
\end{tabular}

\desc

This function should set the Protocol Type Selection (PTS) of a
particular card/slot using the three PTS parameters sent

\begin{itemize}
\item \texttt{Protocol} - 0...14

T=0 to T=14 protocol

\item \texttt{Flags} - Logical OR of possible values to determine which PTS
values to negotiate

\begin{itemize}
\item \texttt{IFD\_NEGOTIATE\_PTS1}
\item \texttt{IFD\_NEGOTIATE\_PTS2}
\item \texttt{IFD\_NEGOTIATE\_PTS3}
\end{itemize}

\item \texttt{PTS1}, \texttt{PTS2}, \texttt{PTS3} - PTS Values

See ISO 7816/EMV documentation.

\end{itemize}

\returns

\begin{tabular}{ll}
\texttt{IFD\_SUCCESS} & Success\\
\texttt{IFD\_ERROR\_PTS\_FAILURE} & Could not set PTS value\\
\texttt{IFD\_COMMUNICATION\_ERROR} & Error has occurred\\
\texttt{IFD\_PROTOCOL\_NOT\_SUPPORTED} & Protocol is not supported\\
\end{tabular}


%---------%---------%---------%---------%---------%---------
\subsection{IFDHPowerICC}

\synopsis
\begin{verbatim}
#include <PCSC/ifdhandler.h>

RESPONSECODE IFDHPowerICC(DWORD Lun,
    DWORD Action,
    PUCHAR Atr,
    PDWORD AtrLength);
\end{verbatim}

\parameters

\begin{tabular}{lll}
\texttt{Lun} & IN & Logical Unit Number\\
\texttt{Action} & IN & Action to be taken\\
\texttt{Atr} & OUT & Answer to Reset (ATR) value of the inserted card\\
\texttt{AtrLength} & INOUT & Length of the ATR\\
\end{tabular}

\desc

This function controls the power and reset signals of the smart card
reader at the particular reader/slot specified by Lun.

\begin{itemize}
\item \texttt{Action} - Action to be taken on the card

\begin{itemize}
\item \texttt{IFD\_POWER\_UP}

Power and reset the card if not done so (store the ATR and return it and
it's length)

\item \texttt{IFD\_POWER\_DOWN}

Power down the card then power up if not done already (\texttt{Atr} and
\texttt{AtrLength} should be zeroed)

\item \texttt{IFD\_RESET}

Perform a quick reset on the card. If the card is not powered power up
the card. (Store and return \texttt{Atr} and \texttt{Length})

\end{itemize}

\item \texttt{Atr} - Answer to Reset of the card

The driver is responsible for caching this value in case
\texttt{IFDHGetCapabilities()} is called requesting the ATR and it's
length. The ATR length should not exceed \texttt{MAX\_ATR\_SIZE}.

\item \texttt{AtrLength} - Length of the Atr

This should not exceed \texttt{MAX\_ATR\_SIZE}.

\end{itemize}

\textbf{Notes:}

Memory cards without an ATR should return \texttt{IFD\_SUCCESS} on reset
but the \texttt{Atr} should be zeroed and the length should be zero
Reset errors should return zero for the \texttt{AtrLength} and return
\texttt{IFD\_ERROR\_POWER\_ACTION}.

\returns

\begin{tabular}{ll}
\texttt{IFD\_SUCCESS} & Success\\
\texttt{IFD\_ERROR\_POWER\_ACTION} & Error powering/resetting card\\
\texttt{IFD\_COMMUNICATION\_ERROR} & An error has occurred\\
\texttt{IFD\_NOT\_SUPPORTED} & Action not supported\\
\end{tabular}


%---------%---------%---------%---------%---------%---------
\subsection{IFDHTransmitToICC}

\synopsis
\begin{verbatim}
#include <PCSC/ifdhandler.h>

RESPONSECODE IFDHTransmitToICC(DWORD Lun,
    SCARD_IO_HEADER SendPci,
    PUCHAR TxBuffer,
    DWORD TxLength,
    PUCHAR RxBuffer,
    PDWORD RxLength,
    PSCARD_IO_HEADER RecvPci);
\end{verbatim}

\parameters

\begin{tabular}{lll}
\texttt{Lun} & IN & Logical Unit Number\\
\texttt{SendPci} & IN & Protocol structure\\
\texttt{TxBuffer} & IN & APDU to be sent\\
\texttt{TxLength} & IN & Length of sent APDU\\
\texttt{RxBuffer} & OUT & APDU response\\
\texttt{RxLength} & INOUT & Length of APDU response\\
\texttt{RecvPci} & INOUT & Receive protocol structure
\end{tabular}

\desc

This function performs an APDU exchange with the card/slot specified by
\texttt{Lun}. The driver is responsible for performing any protocol
specific exchanges such as T=0, 1, etc. differences. Calling this function
will abstract all protocol differences.

\begin{itemize}
\item \texttt{SendPci} - contains two structure members

\begin{itemize}
\item \texttt{Protocol} - 0, 1, ... 14

T=0 ... T=14

\item \texttt{Length} \tab - Not used.
\end{itemize}

\item \texttt{TxBuffer} \tab - Transmit APDU

example: \verb+"\x00\xA4\x00\x00\x02\x3F\x00"+

\item \texttt{TxLength} \tab - Length of this buffer
\item \texttt{RxBuffer} \tab - Receive APDU

example: \verb+"\x61\x14"+

\item \texttt{RxLength} \tab - Length of the received APDU

This function will be passed the size of the buffer of \texttt{RxBuffer}
and this function is responsible for setting this to the length of the
received APDU response. This should be ZERO on all errors. The resource
manager will take responsibility of zeroing out any temporary APDU
buffers for security reasons.

\item \texttt{RecvPci} - contains two structure members

\begin{itemize}
\item \texttt{Protocol} - 0, 1, ... 14

T=0 ... T=14

\item \texttt{Length} - Not used.
\end{itemize}
\end{itemize}

\textbf{Notes:}

The driver is responsible for knowing what type of card it has.  If the
current slot/card contains a memory card then this command should ignore
the \texttt{Protocol} and use the MCT style commands for support for
these style cards and transmit them appropriately. If your reader does
not support memory cards or you don't want to implement this
functionality, then ignore this.

\texttt{RxLength} should be set to zero on error.

The driver is \emph{not} responsible for doing an automatic Get Response
command for received buffers containing \texttt{61 XX}.

\returns

\begin{tabular}{ll}
\texttt{IFD\_SUCCESS} & Success\\
\texttt{IFD\_COMMUNICATION\_ERROR} & An error has occurred\\
\texttt{IFD\_RESPONSE\_TIMEOUT} & The response timed out\\
\texttt{IFD\_ICC\_NOT\_PRESENT} & ICC is not present\\
\texttt{IFD\_PROTOCOL\_NOT\_SUPPORTED} & Protocol is not supported
\end{tabular}


%---------%---------%---------%---------%---------%---------
\subsection{IFDHControl}
\label{IFDHControl}

\synopsis
\begin{verbatim}
#include <PCSC/ifdhandler.h>

RESPONSECODE IFDHControl(DWORD Lun,
    DWORD dwControlCode,
    PUCHAR TxBuffer,
    DWORD TxLength,
    PUCHAR RxBuffer,
    DWORD RxLength,
    PDWORD pdwBytesReturned);
\end{verbatim}

\parameters

\begin{tabular}{lll}
\texttt{Lun} & IN & Logical Unit Number\\
\texttt{dwControlCode} & IN & Control code for the operation\\
\texttt{TxBuffer} & IN & Bytes to be sent\\
\texttt{TxLength} & IN & Length of sent bytes\\
\texttt{RxBuffer} & OUT & Response\\
\texttt{RxLength} & IN & Length of response buffer\\
\texttt{pdwBytesReturned} & OUT & Length of response \\
\end{tabular}

\desc

This function performs a data exchange with the reader (not the card)
specified by \texttt{Lun}. It is responsible for abstracting
functionality such as PIN pads, biometrics, LCD panels, etc.  You should
follow the MCT and CTBCS specifications for a list of accepted commands
to implement. This function is fully voluntary and does not have to be
implemented unless you want extended functionality.

\begin{itemize}
\item \texttt{dwControlCode} - Control code for the operation

This value identifies the specific operation to be performed. This value
is driver specific.

\item \texttt{TxBuffer} - Transmit data
\item \texttt{TxLength} - Length of this buffer
\item \texttt{RxBuffer} - Receive data
\item \texttt{RxLength} - Length of the response buffer

\item \texttt{pdwBytesReturned} - Length of response

This function will be passed the length of the buffer \texttt{RxBuffer}
in \texttt{RxLength} and it must set the length of the received data in
\texttt{pdwBytesReturned}.

\end{itemize}

\textbf{Notes:}

\texttt{*pdwBytesReturned} should be set to zero on error.

\returns

\begin{tabular}{ll}
\texttt{IFD\_SUCCESS} & Success\\
\texttt{IFD\_COMMUNICATION\_ERROR} & An error has occurred\\
\texttt{IFD\_RESPONSE\_TIMEOUT} & The response timed out
\end{tabular}


%---------%---------%---------%---------%---------%---------
\subsection{IFDHICCPresence}

\synopsis
\begin{verbatim}
#include <PCSC/ifdhandler.h>

RESPONSECODE IFDHICCPresence(DWORD Lun);
\end{verbatim}

\parameters

\begin{tabular}{lll}
\texttt{Lun} & IN & Logical Unit Number
\end{tabular}

\desc

This function returns the status of the card inserted in the reader/slot
specified by \texttt{Lun}. In cases where the device supports
asynchronous card insertion/removal detection, it is advised that the
driver manages this through a thread so the driver does not have to send
and receive a command each time this function is called.

\returns

\begin{tabular}{ll}
\texttt{IFD\_ICC\_PRESENT} & ICC is present\\
\texttt{IFD\_ICC\_NOT\_PRESENT} & ICC is not present\\
\texttt{IFD\_COMMUNICATION\_ERROR} & An error has occurred
\end{tabular}


%---------%---------%---------%---------%---------%---------%---------
\section{API provided by pcsc-lite}

pcsc-lite also provides some API to ease the development of the driver.


%---------%---------%---------%---------%---------%---------
\subsection{LTPBundleFindValueWithKey}

\synopsis
\begin{verbatim}
#include <PCSC/parser.h>

int LTPBundleFindValueWithKey(char *fileName,
    char *tokenKey,
    char *tokenValue,
    int tokenIndice);
\end{verbatim}

\parameters

\begin{tabular}{lll}
\texttt{fileName} & IN & configuration file name \\
\texttt{tokenKey} & IN & searched token string \\
\texttt{tokenValue} & OUT & found token value \\
\texttt{tokenIndice} & IN & indice of the desired token in an array \\
\end{tabular}

\desc

This function is used to parse the \texttt{Info.plist} configuration
file. The \texttt{tokenIndice} field is used to get the n-th element of
an array.

\returns

\begin{tabular}{ll}
\texttt{-1} & Error\\
\texttt{0} & No error \\
\texttt{1} & Configuration file not found \\
\end{tabular}


%---------%---------%---------%---------%---------%---------
\subsection{debug\_msg}

\synopsis
\begin{verbatim}
#include <PCSC/debuglog.h>

void debug_msg(const char *fmt,
    ...);
\end{verbatim}

\parameters

\begin{tabular}{lll}
\texttt{fmt} & IN & format string as in \texttt{printf()} \\
\texttt{...} & IN & optionnal parameters as in \texttt{printf()} \\
\end{tabular}

\desc

This function is used by the driver to send debug information. The
advantage of using the same debug function as pcsc-lite is that you also
benefit from the debug redirection provided by pcsc-lite. For example
you can use \texttt{pcscd --foreground} to start \texttt{pcscd} and keep
it in the forground with the debug sent to stderr. You will then get
\texttt{pcscd} and the driver' debug messages in the same place.

\example
\begin{verbatim}
#define DebugLogB(fmt, data) debug_msg("%s:%d:%s " fmt, \
    __FILE__, __LINE__, __FUNCTION__, data)

DebugLogB("received bytes: %d", r);
\end{verbatim}


%---------%---------%---------%---------%---------%---------
\subsection{debug\_xxd}

\synopsis
\begin{verbatim}
#include <PCSC/debuglog.h>

void debug_xxd(const char *msg,
    const unsigned char *buffer,
    const int size);
\end{verbatim}

\parameters

\begin{tabular}{lll}
\texttt{msg} & IN & text string \\
\texttt{buffer} & IN & buffer you want to dump in hex \\
\texttt{size} & IN & size of the buffer \\
\end{tabular}

\desc

Same idea as \texttt{debug\_msg()} put print the hex dump of a buffer.

\example
\begin{verbatim}
debug_xxd("received frame: ", buff, buff_size);
\end{verbatim}


%---------%---------%---------%---------%---------%---------%---------
\section{API changes}

The IFD handler API changed over the time.

If the driver provides a \texttt{IFDHCreateChannelByName()} function is
supposed to use API v3.0. Otherwise it is used with API v2.0.


%---------%---------%---------%---------%---------%---------
\subsection{API version 2.0}

\begin{itemize}
\item \texttt{DEVICENAME} in \texttt{reader.conf} is not used. 

\item \texttt{IFDHControl()} API was:
\begin{verbatim}
RESPONSECODE IFDHControl(DWORD Lun,
    PUCHAR TxBuffer,
    DWORD TxLength,
    PUCHAR RxBuffer,
    PDWORD RxLength);
\end{verbatim}

\end{itemize}


%---------%---------%---------%---------%---------%---------
\subsection{API version 3.0}

\begin{itemize}
\item introduction of \texttt{IFDHCreateChannelByName()}.

For serial drivers, \texttt{CHANNELID} is no more used and
\texttt{DEVICENAME} is used instead.

For USB drivers the device name if \verb+usb:%04x/%04x:libusb:%s:%s+.
See \ref{IFDHCreateChannelByName}.

\item \texttt{IFDHControl()} API changed

See \ref{IFDHControl}.

\end{itemize}


%---------%---------%---------%---------%---------%---------
\bibliographystyle{plain}
\bibliography{pcsc-lite}


\end{document}

